% Options for packages loaded elsewhere
\PassOptionsToPackage{unicode}{hyperref}
\PassOptionsToPackage{hyphens}{url}
%
\documentclass[
]{article}
\usepackage{amsmath,amssymb}
\usepackage{lmodern}
\usepackage{iftex}
\ifPDFTeX
  \usepackage[T1]{fontenc}
  \usepackage[utf8]{inputenc}
  \usepackage{textcomp} % provide euro and other symbols
\else % if luatex or xetex
  \usepackage{unicode-math}
  \defaultfontfeatures{Scale=MatchLowercase}
  \defaultfontfeatures[\rmfamily]{Ligatures=TeX,Scale=1}
\fi
% Use upquote if available, for straight quotes in verbatim environments
\IfFileExists{upquote.sty}{\usepackage{upquote}}{}
\IfFileExists{microtype.sty}{% use microtype if available
  \usepackage[]{microtype}
  \UseMicrotypeSet[protrusion]{basicmath} % disable protrusion for tt fonts
}{}
\makeatletter
\@ifundefined{KOMAClassName}{% if non-KOMA class
  \IfFileExists{parskip.sty}{%
    \usepackage{parskip}
  }{% else
    \setlength{\parindent}{0pt}
    \setlength{\parskip}{6pt plus 2pt minus 1pt}}
}{% if KOMA class
  \KOMAoptions{parskip=half}}
\makeatother
\usepackage{xcolor}
\usepackage[margin=1in]{geometry}
\usepackage{color}
\usepackage{fancyvrb}
\newcommand{\VerbBar}{|}
\newcommand{\VERB}{\Verb[commandchars=\\\{\}]}
\DefineVerbatimEnvironment{Highlighting}{Verbatim}{commandchars=\\\{\}}
% Add ',fontsize=\small' for more characters per line
\usepackage{framed}
\definecolor{shadecolor}{RGB}{248,248,248}
\newenvironment{Shaded}{\begin{snugshade}}{\end{snugshade}}
\newcommand{\AlertTok}[1]{\textcolor[rgb]{0.94,0.16,0.16}{#1}}
\newcommand{\AnnotationTok}[1]{\textcolor[rgb]{0.56,0.35,0.01}{\textbf{\textit{#1}}}}
\newcommand{\AttributeTok}[1]{\textcolor[rgb]{0.77,0.63,0.00}{#1}}
\newcommand{\BaseNTok}[1]{\textcolor[rgb]{0.00,0.00,0.81}{#1}}
\newcommand{\BuiltInTok}[1]{#1}
\newcommand{\CharTok}[1]{\textcolor[rgb]{0.31,0.60,0.02}{#1}}
\newcommand{\CommentTok}[1]{\textcolor[rgb]{0.56,0.35,0.01}{\textit{#1}}}
\newcommand{\CommentVarTok}[1]{\textcolor[rgb]{0.56,0.35,0.01}{\textbf{\textit{#1}}}}
\newcommand{\ConstantTok}[1]{\textcolor[rgb]{0.00,0.00,0.00}{#1}}
\newcommand{\ControlFlowTok}[1]{\textcolor[rgb]{0.13,0.29,0.53}{\textbf{#1}}}
\newcommand{\DataTypeTok}[1]{\textcolor[rgb]{0.13,0.29,0.53}{#1}}
\newcommand{\DecValTok}[1]{\textcolor[rgb]{0.00,0.00,0.81}{#1}}
\newcommand{\DocumentationTok}[1]{\textcolor[rgb]{0.56,0.35,0.01}{\textbf{\textit{#1}}}}
\newcommand{\ErrorTok}[1]{\textcolor[rgb]{0.64,0.00,0.00}{\textbf{#1}}}
\newcommand{\ExtensionTok}[1]{#1}
\newcommand{\FloatTok}[1]{\textcolor[rgb]{0.00,0.00,0.81}{#1}}
\newcommand{\FunctionTok}[1]{\textcolor[rgb]{0.00,0.00,0.00}{#1}}
\newcommand{\ImportTok}[1]{#1}
\newcommand{\InformationTok}[1]{\textcolor[rgb]{0.56,0.35,0.01}{\textbf{\textit{#1}}}}
\newcommand{\KeywordTok}[1]{\textcolor[rgb]{0.13,0.29,0.53}{\textbf{#1}}}
\newcommand{\NormalTok}[1]{#1}
\newcommand{\OperatorTok}[1]{\textcolor[rgb]{0.81,0.36,0.00}{\textbf{#1}}}
\newcommand{\OtherTok}[1]{\textcolor[rgb]{0.56,0.35,0.01}{#1}}
\newcommand{\PreprocessorTok}[1]{\textcolor[rgb]{0.56,0.35,0.01}{\textit{#1}}}
\newcommand{\RegionMarkerTok}[1]{#1}
\newcommand{\SpecialCharTok}[1]{\textcolor[rgb]{0.00,0.00,0.00}{#1}}
\newcommand{\SpecialStringTok}[1]{\textcolor[rgb]{0.31,0.60,0.02}{#1}}
\newcommand{\StringTok}[1]{\textcolor[rgb]{0.31,0.60,0.02}{#1}}
\newcommand{\VariableTok}[1]{\textcolor[rgb]{0.00,0.00,0.00}{#1}}
\newcommand{\VerbatimStringTok}[1]{\textcolor[rgb]{0.31,0.60,0.02}{#1}}
\newcommand{\WarningTok}[1]{\textcolor[rgb]{0.56,0.35,0.01}{\textbf{\textit{#1}}}}
\usepackage{longtable,booktabs,array}
\usepackage{calc} % for calculating minipage widths
% Correct order of tables after \paragraph or \subparagraph
\usepackage{etoolbox}
\makeatletter
\patchcmd\longtable{\par}{\if@noskipsec\mbox{}\fi\par}{}{}
\makeatother
% Allow footnotes in longtable head/foot
\IfFileExists{footnotehyper.sty}{\usepackage{footnotehyper}}{\usepackage{footnote}}
\makesavenoteenv{longtable}
\usepackage{graphicx}
\makeatletter
\def\maxwidth{\ifdim\Gin@nat@width>\linewidth\linewidth\else\Gin@nat@width\fi}
\def\maxheight{\ifdim\Gin@nat@height>\textheight\textheight\else\Gin@nat@height\fi}
\makeatother
% Scale images if necessary, so that they will not overflow the page
% margins by default, and it is still possible to overwrite the defaults
% using explicit options in \includegraphics[width, height, ...]{}
\setkeys{Gin}{width=\maxwidth,height=\maxheight,keepaspectratio}
% Set default figure placement to htbp
\makeatletter
\def\fps@figure{htbp}
\makeatother
\setlength{\emergencystretch}{3em} % prevent overfull lines
\providecommand{\tightlist}{%
  \setlength{\itemsep}{0pt}\setlength{\parskip}{0pt}}
\setcounter{secnumdepth}{-\maxdimen} % remove section numbering
\ifLuaTeX
  \usepackage{selnolig}  % disable illegal ligatures
\fi
\IfFileExists{bookmark.sty}{\usepackage{bookmark}}{\usepackage{hyperref}}
\IfFileExists{xurl.sty}{\usepackage{xurl}}{} % add URL line breaks if available
\urlstyle{same} % disable monospaced font for URLs
\hypersetup{
  pdftitle={EDA on Furniture store Transaction Dataset},
  pdfauthor={Aashish Telgote},
  hidelinks,
  pdfcreator={LaTeX via pandoc}}

\title{EDA on Furniture store Transaction Dataset}
\author{Aashish Telgote}
\date{2023-02-10}

\begin{document}
\maketitle

\hypertarget{introduction}{%
\subsection{Introduction}\label{introduction}}

\hypertarget{in-this-project-we-will-do-little-cleaning-and-exploratory-data-analysis-on-furniture-transactions-data.-i-used-a-small-dataset-in-this-project-but-you-can-follow-similar-steps-on-a-larger-dataset-and-you-will-still-get-accurate-results.}{%
\subsubsection{In this project, we will do little cleaning and
exploratory data analysis on ``Furniture Transactions Data''. I used a
small dataset in this project, but you can follow similar steps on a
larger dataset and you will still get accurate
results.}\label{in-this-project-we-will-do-little-cleaning-and-exploratory-data-analysis-on-furniture-transactions-data.-i-used-a-small-dataset-in-this-project-but-you-can-follow-similar-steps-on-a-larger-dataset-and-you-will-still-get-accurate-results.}}

\hypertarget{ask-phase}{%
\subsection{ASK PHASE}\label{ask-phase}}

\hypertarget{we-will-find-answers-to-certain-questions-that-will-be-useful-for-business-decisions.}{%
\subsubsection{We will find answers to certain questions that will be
useful for business
decisions.}\label{we-will-find-answers-to-certain-questions-that-will-be-useful-for-business-decisions.}}

\hypertarget{what-is-the-total-revenue-generated-by-each-product}{%
\paragraph{1. What is the total revenue generated by each
product?}\label{what-is-the-total-revenue-generated-by-each-product}}

\hypertarget{how-many-units-of-each-product-were-sold}{%
\paragraph{2. How many units of each product were
sold?}\label{how-many-units-of-each-product-were-sold}}

\hypertarget{from-which-customer-have-we-made-the-most-revenue}{%
\paragraph{3. From which customer have we made the most
revenue?}\label{from-which-customer-have-we-made-the-most-revenue}}

\hypertarget{how-many-products-did-each-customer-buy}{%
\paragraph{4. How many products did each customer
buy?}\label{how-many-products-did-each-customer-buy}}

\hypertarget{which-color-is-most-preferred-by-customers-in-product-named-fan}{%
\paragraph{5. Which color is most preferred by customers in product
named
``Fan''?}\label{which-color-is-most-preferred-by-customers-in-product-named-fan}}

\hypertarget{which-color-is-most-preferred-by-customers-in-product-named-couch}{%
\paragraph{6. Which color is most preferred by customers in product
named
``Couch''?}\label{which-color-is-most-preferred-by-customers-in-product-named-couch}}

\hypertarget{which-color-is-most-preferred-by-customers-in-product-named-rug}{%
\paragraph{7. Which color is most preferred by customers in product
named
``Rug''?}\label{which-color-is-most-preferred-by-customers-in-product-named-rug}}

\hypertarget{which-color-is-most-preferred-by-customers-in-product-named-desk}{%
\paragraph{8. Which color is most preferred by customers in product
named
``Desk''?}\label{which-color-is-most-preferred-by-customers-in-product-named-desk}}

\hypertarget{prepare-phase}{%
\subsection{PREPARE PHASE :}\label{prepare-phase}}

\hypertarget{this-is-a-practice-dataset-from-google-data-analytics-professional-specialization-course.}{%
\paragraph{\texorpdfstring{This is a practice dataset from
\textbf{Google data analytics professional specialization
course}.}{This is a practice dataset from Google data analytics professional specialization course.}}\label{this-is-a-practice-dataset-from-google-data-analytics-professional-specialization-course.}}

\hypertarget{to-view-the-dataset-click-here}{%
\paragraph{\texorpdfstring{To view the dataset,
\href{https://drive.google.com/file/d/1iLwrjh05klmLS4tDVkU97bSI3J14CQ7X/view?usp=share_link}{click
here}}{To view the dataset, click here}}\label{to-view-the-dataset-click-here}}

\hypertarget{now-lets-install-some-required-r-packages-to-start-our-work.}{%
\subsubsection{Now, let's install some required R packages to start our
work.}\label{now-lets-install-some-required-r-packages-to-start-our-work.}}

\hypertarget{we-will-start-with-tidyverse-package.}{%
\paragraph{We will start with tidyverse
package.}\label{we-will-start-with-tidyverse-package.}}

\hypertarget{tidyverse-is-a-collection-of-packages-in-r-with-a-common-design-philosophy-for-data-manipulation-exploration-and-visualization.}{%
\paragraph{Tidyverse is a collection of packages in R with a common
design philosophy for data manipulation, exploration and
visualization.}\label{tidyverse-is-a-collection-of-packages-in-r-with-a-common-design-philosophy-for-data-manipulation-exploration-and-visualization.}}

\hypertarget{usually-tidyverse-package-is-all-we-need-for-data-analysis.}{%
\paragraph{Usually, Tidyverse package is all we need for data
analysis.}\label{usually-tidyverse-package-is-all-we-need-for-data-analysis.}}

\begin{Shaded}
\begin{Highlighting}[]
\FunctionTok{install.packages}\NormalTok{(}\StringTok{"tidyverse"}\NormalTok{)}
\end{Highlighting}
\end{Shaded}

\begin{verbatim}
## Installing package into '/cloud/lib/x86_64-pc-linux-gnu-library/4.2'
## (as 'lib' is unspecified)
\end{verbatim}

\begin{Shaded}
\begin{Highlighting}[]
\FunctionTok{library}\NormalTok{(}\StringTok{"tidyverse"}\NormalTok{)}
\end{Highlighting}
\end{Shaded}

\begin{verbatim}
## -- Attaching core tidyverse packages ------------------------ tidyverse 2.0.0 --
## v dplyr     1.1.0     v readr     2.1.4
## v forcats   1.0.0     v stringr   1.5.0
## v ggplot2   3.4.2     v tibble    3.1.8
## v lubridate 1.9.2     v tidyr     1.3.0
## v purrr     1.0.1     
## -- Conflicts ------------------------------------------ tidyverse_conflicts() --
## x dplyr::filter() masks stats::filter()
## x dplyr::lag()    masks stats::lag()
## i Use the ]8;;http://conflicted.r-lib.org/conflicted package]8;; to force all conflicts to become errors
\end{verbatim}

\begin{Shaded}
\begin{Highlighting}[]
\FunctionTok{library}\NormalTok{(}\StringTok{"readr"}\NormalTok{)}
\end{Highlighting}
\end{Shaded}

\hypertarget{lets-import-our-dataset-in-rmarkdown.-so-that-we-can-knit-it-to-create-a-final-document}{%
\subsubsection{Let's import our dataset in Rmarkdown. So that, we can
knit it to create a final
document}\label{lets-import-our-dataset-in-rmarkdown.-so-that-we-can-knit-it-to-create-a-final-document}}

\begin{Shaded}
\begin{Highlighting}[]
\NormalTok{Store\_Transactions }\OtherTok{\textless{}{-}} \FunctionTok{read.csv}\NormalTok{(}\StringTok{"Store\_Transactions.csv"}\NormalTok{, }\AttributeTok{header =} \ConstantTok{TRUE}\NormalTok{, }\AttributeTok{sep =} \StringTok{\textquotesingle{},\textquotesingle{}}\NormalTok{)}
\end{Highlighting}
\end{Shaded}

\hypertarget{now-we-will-install-and-load-janitor-package.-it-has-functions-for-cleaning-data.}{%
\subsubsection{Now, we will install and load ``Janitor package''. It has
functions for cleaning
data.}\label{now-we-will-install-and-load-janitor-package.-it-has-functions-for-cleaning-data.}}

\begin{Shaded}
\begin{Highlighting}[]
\FunctionTok{install.packages}\NormalTok{(}\StringTok{"janitor"}\NormalTok{)}
\end{Highlighting}
\end{Shaded}

\begin{verbatim}
## Installing package into '/cloud/lib/x86_64-pc-linux-gnu-library/4.2'
## (as 'lib' is unspecified)
\end{verbatim}

\begin{Shaded}
\begin{Highlighting}[]
\FunctionTok{library}\NormalTok{(}\StringTok{"janitor"}\NormalTok{)}
\end{Highlighting}
\end{Shaded}

\begin{verbatim}
## 
## Attaching package: 'janitor'
\end{verbatim}

\begin{verbatim}
## The following objects are masked from 'package:stats':
## 
##     chisq.test, fisher.test
\end{verbatim}

\hypertarget{now-we-will-install-dplyr-package-as-will-be-using-some-of-its-functions.}{%
\subsubsection{Now, we will install ``dplyr package'' as will be using
some of it's
functions.}\label{now-we-will-install-dplyr-package-as-will-be-using-some-of-its-functions.}}

\begin{Shaded}
\begin{Highlighting}[]
\FunctionTok{install.packages}\NormalTok{(}\StringTok{"dplyr"}\NormalTok{)}
\end{Highlighting}
\end{Shaded}

\begin{verbatim}
## Installing package into '/cloud/lib/x86_64-pc-linux-gnu-library/4.2'
## (as 'lib' is unspecified)
\end{verbatim}

\begin{verbatim}
## also installing the dependencies 'pillar', 'tibble', 'vctrs'
\end{verbatim}

\begin{Shaded}
\begin{Highlighting}[]
\FunctionTok{library}\NormalTok{(}\StringTok{"dplyr"}\NormalTok{)}
\end{Highlighting}
\end{Shaded}

\hypertarget{now-lets-install-skimr-package.-it-lets-us-summarize-the-data-and-skim-through-it-quickly.}{%
\subsubsection{Now, lets install ``skimr package''. It let's us
summarize the data and skim through it
quickly.}\label{now-lets-install-skimr-package.-it-lets-us-summarize-the-data-and-skim-through-it-quickly.}}

\begin{Shaded}
\begin{Highlighting}[]
\FunctionTok{install.packages}\NormalTok{(}\StringTok{"skimr"}\NormalTok{)}
\end{Highlighting}
\end{Shaded}

\begin{verbatim}
## Installing package into '/cloud/lib/x86_64-pc-linux-gnu-library/4.2'
## (as 'lib' is unspecified)
\end{verbatim}

\begin{Shaded}
\begin{Highlighting}[]
\FunctionTok{library}\NormalTok{(}\StringTok{"skimr"}\NormalTok{)}
\end{Highlighting}
\end{Shaded}

\hypertarget{now-lets-see-the-summary-and-basic-statistics-of-the-dataset}{%
\subsubsection{Now, let's see the summary and basic statistics of the
dataset}\label{now-lets-see-the-summary-and-basic-statistics-of-the-dataset}}

\begin{Shaded}
\begin{Highlighting}[]
\FunctionTok{skim\_without\_charts}\NormalTok{(Store\_Transactions)}
\end{Highlighting}
\end{Shaded}

\begin{longtable}[]{@{}ll@{}}
\caption{Data summary}\tabularnewline
\toprule()
\endhead
Name & Store\_Transactions \\
Number of rows & 29 \\
Number of columns & 10 \\
\_\_\_\_\_\_\_\_\_\_\_\_\_\_\_\_\_\_\_\_\_\_\_ & \\
Column type frequency: & \\
character & 5 \\
numeric & 5 \\
\_\_\_\_\_\_\_\_\_\_\_\_\_\_\_\_\_\_\_\_\_\_\_\_ & \\
Group variables & None \\
\bottomrule()
\end{longtable}

\textbf{Variable type: character}

\begin{longtable}[]{@{}
  >{\raggedright\arraybackslash}p{(\columnwidth - 14\tabcolsep) * \real{0.1944}}
  >{\raggedleft\arraybackslash}p{(\columnwidth - 14\tabcolsep) * \real{0.1389}}
  >{\raggedleft\arraybackslash}p{(\columnwidth - 14\tabcolsep) * \real{0.1944}}
  >{\raggedleft\arraybackslash}p{(\columnwidth - 14\tabcolsep) * \real{0.0556}}
  >{\raggedleft\arraybackslash}p{(\columnwidth - 14\tabcolsep) * \real{0.0556}}
  >{\raggedleft\arraybackslash}p{(\columnwidth - 14\tabcolsep) * \real{0.0833}}
  >{\raggedleft\arraybackslash}p{(\columnwidth - 14\tabcolsep) * \real{0.1250}}
  >{\raggedleft\arraybackslash}p{(\columnwidth - 14\tabcolsep) * \real{0.1528}}@{}}
\toprule()
\begin{minipage}[b]{\linewidth}\raggedright
skim\_variable
\end{minipage} & \begin{minipage}[b]{\linewidth}\raggedleft
n\_missing
\end{minipage} & \begin{minipage}[b]{\linewidth}\raggedleft
complete\_rate
\end{minipage} & \begin{minipage}[b]{\linewidth}\raggedleft
min
\end{minipage} & \begin{minipage}[b]{\linewidth}\raggedleft
max
\end{minipage} & \begin{minipage}[b]{\linewidth}\raggedleft
empty
\end{minipage} & \begin{minipage}[b]{\linewidth}\raggedleft
n\_unique
\end{minipage} & \begin{minipage}[b]{\linewidth}\raggedleft
whitespace
\end{minipage} \\
\midrule()
\endhead
date & 0 & 1 & 15 & 15 & 0 & 24 & 0 \\
product & 0 & 1 & 0 & 8 & 2 & 11 & 0 \\
product\_code & 0 & 1 & 8 & 8 & 0 & 12 & 0 \\
product\_color & 0 & 1 & 4 & 6 & 0 & 9 & 0 \\
revenue & 0 & 1 & 7 & 10 & 0 & 15 & 0 \\
\bottomrule()
\end{longtable}

\textbf{Variable type: numeric}

\begin{longtable}[]{@{}
  >{\raggedright\arraybackslash}p{(\columnwidth - 18\tabcolsep) * \real{0.1579}}
  >{\raggedleft\arraybackslash}p{(\columnwidth - 18\tabcolsep) * \real{0.1053}}
  >{\raggedleft\arraybackslash}p{(\columnwidth - 18\tabcolsep) * \real{0.1474}}
  >{\raggedleft\arraybackslash}p{(\columnwidth - 18\tabcolsep) * \real{0.0947}}
  >{\raggedleft\arraybackslash}p{(\columnwidth - 18\tabcolsep) * \real{0.0947}}
  >{\raggedleft\arraybackslash}p{(\columnwidth - 18\tabcolsep) * \real{0.0842}}
  >{\raggedleft\arraybackslash}p{(\columnwidth - 18\tabcolsep) * \real{0.0947}}
  >{\raggedleft\arraybackslash}p{(\columnwidth - 18\tabcolsep) * \real{0.0947}}
  >{\raggedleft\arraybackslash}p{(\columnwidth - 18\tabcolsep) * \real{0.0632}}
  >{\raggedleft\arraybackslash}p{(\columnwidth - 18\tabcolsep) * \real{0.0632}}@{}}
\toprule()
\begin{minipage}[b]{\linewidth}\raggedright
skim\_variable
\end{minipage} & \begin{minipage}[b]{\linewidth}\raggedleft
n\_missing
\end{minipage} & \begin{minipage}[b]{\linewidth}\raggedleft
complete\_rate
\end{minipage} & \begin{minipage}[b]{\linewidth}\raggedleft
mean
\end{minipage} & \begin{minipage}[b]{\linewidth}\raggedleft
sd
\end{minipage} & \begin{minipage}[b]{\linewidth}\raggedleft
p0
\end{minipage} & \begin{minipage}[b]{\linewidth}\raggedleft
p25
\end{minipage} & \begin{minipage}[b]{\linewidth}\raggedleft
p50
\end{minipage} & \begin{minipage}[b]{\linewidth}\raggedleft
p75
\end{minipage} & \begin{minipage}[b]{\linewidth}\raggedleft
p100
\end{minipage} \\
\midrule()
\endhead
transaction\_id & 0 & 1 & 27283.28 & 15388.50 & 1675.00 & 12560.00 &
24785.00 & 44700 & 49430 \\
customer\_id & 0 & 1 & 5456.66 & 3077.70 & 335.00 & 2512.00 & 4957.00 &
8940 & 9886 \\
product\_price & 0 & 1 & 413.39 & 429.06 & 9.99 & 58.89 & 169.95 & 1000
& 1000 \\
purchase\_size & 0 & 1 & 1.45 & 0.91 & 1.00 & 1.00 & 1.00 & 2 & 5 \\
purchase\_price & 0 & 1 & 434.64 & 414.52 & 13.99 & 89.85 & 234.50 &
1000 & 1000 \\
\bottomrule()
\end{longtable}

\hypertarget{lets-see-the-structure-of-the-dataset-and-datatype-of-each-column.}{%
\subsubsection{Let's see the structure of the dataset and datatype of
each
column.}\label{lets-see-the-structure-of-the-dataset-and-datatype-of-each-column.}}

\begin{Shaded}
\begin{Highlighting}[]
\FunctionTok{str}\NormalTok{(Store\_Transactions)}
\end{Highlighting}
\end{Shaded}

\begin{verbatim}
## 'data.frame':    29 obs. of  10 variables:
##  $ date          : chr  "29/08/2020 0:00" "01/05/2020 0:00" "12/12/2020 0:00" "16/02/2020 0:00" ...
##  $ transaction_id: int  9900 12315 9890 46915 44700 44700 12560 9640 22620 49430 ...
##  $ customer_id   : int  1980 2463 1978 9383 8940 8940 2512 1928 4524 9886 ...
##  $ product       : chr  "fan" "fan" "fan" "fan" ...
##  $ product_code  : chr  "SKU83503" "SKU83503" "SKU83503" "SKU83503" ...
##  $ product_color : chr  "brass" "brass" "white" "black" ...
##  $ product_price : num  14 14 14 14 14 ...
##  $ purchase_size : int  2 2 1 1 2 5 1 1 1 1 ...
##  $ purchase_price: num  28 28 14 14 28 ...
##  $ revenue       : chr  "$27.98 " "$27.98 " "$13.99 " "$13.99 " ...
\end{verbatim}

\hypertarget{now-we-will-take-a-glimpse-of-the-dataset}{%
\subsubsection{Now, we will take a glimpse of the
dataset}\label{now-we-will-take-a-glimpse-of-the-dataset}}

\begin{Shaded}
\begin{Highlighting}[]
\FunctionTok{glimpse}\NormalTok{(Store\_Transactions)}
\end{Highlighting}
\end{Shaded}

\begin{verbatim}
## Rows: 29
## Columns: 10
## $ date           <chr> "29/08/2020 0:00", "01/05/2020 0:00", "12/12/2020 0:00"~
## $ transaction_id <int> 9900, 12315, 9890, 46915, 44700, 44700, 12560, 9640, 22~
## $ customer_id    <int> 1980, 2463, 1978, 9383, 8940, 8940, 2512, 1928, 4524, 9~
## $ product        <chr> "fan", "fan", "fan", "fan", "fan", "lamp", "bed", "couc~
## $ product_code   <chr> "SKU83503", "SKU83503", "SKU83503", "SKU83503", "SKU835~
## $ product_color  <chr> "brass", "brass", "white", "black", "brass", "brass", "~
## $ product_price  <dbl> 13.99, 13.99, 13.99, 13.99, 13.99, 45.99, 799.99, 1000.~
## $ purchase_size  <int> 2, 2, 1, 1, 2, 5, 1, 1, 1, 1, 1, 1, 1, 1, 1, 1, 3, 2, 1~
## $ purchase_price <dbl> 27.980, 27.980, 13.990, 13.990, 27.980, 160.965, 799.99~
## $ revenue        <chr> "$27.98 ", "$27.98 ", "$13.99 ", "$13.99 ", "$27.98 ", ~
\end{verbatim}

\hypertarget{now-if-we-want-we-can-only-check-all-the-column-names}{%
\subsubsection{Now, if we want we can only check all the column
names}\label{now-if-we-want-we-can-only-check-all-the-column-names}}

\begin{Shaded}
\begin{Highlighting}[]
\FunctionTok{colnames}\NormalTok{(Store\_Transactions)}
\end{Highlighting}
\end{Shaded}

\begin{verbatim}
##  [1] "date"           "transaction_id" "customer_id"    "product"       
##  [5] "product_code"   "product_color"  "product_price"  "purchase_size" 
##  [9] "purchase_price" "revenue"
\end{verbatim}

\hypertarget{lets-preview-the-dataset-to-know-how-it-looks-in-tabular-format.}{%
\subsubsection{Let's preview the dataset to know how it looks in tabular
format.}\label{lets-preview-the-dataset-to-know-how-it-looks-in-tabular-format.}}

\begin{Shaded}
\begin{Highlighting}[]
\FunctionTok{head}\NormalTok{(Store\_Transactions)}
\end{Highlighting}
\end{Shaded}

\begin{verbatim}
##              date transaction_id customer_id product product_code product_color
## 1 29/08/2020 0:00           9900        1980     fan     SKU83503         brass
## 2 01/05/2020 0:00          12315        2463     fan     SKU83503         brass
## 3 12/12/2020 0:00           9890        1978     fan     SKU83503         white
## 4 16/02/2020 0:00          46915        9383     fan     SKU83503         black
## 5 28/12/2020 0:00          44700        8940     fan     SKU83503         brass
## 6 28/12/2020 0:00          44700        8940    lamp     SKU95363         brass
##   product_price purchase_size purchase_price  revenue
## 1         13.99             2         27.980  $27.98 
## 2         13.99             2         27.980  $27.98 
## 3         13.99             1         13.990  $13.99 
## 4         13.99             1         13.990  $13.99 
## 5         13.99             2         27.980  $27.98 
## 6         45.99             5        160.965 $229.95
\end{verbatim}

\hypertarget{process-phase}{%
\subsection{PROCESS PHASE}\label{process-phase}}

\hypertarget{in-this-phase-we-will-do-some-data-cleaning.}{%
\subsubsection{In this phase, we will do some data
cleaning.}\label{in-this-phase-we-will-do-some-data-cleaning.}}

\hypertarget{lets-rename-the-product-and-purchase-size-column-to-product_name-and-units_purchased-respectively-for-better-understanding-of-underlying-data-in-the-column.}{%
\paragraph{Let's rename the ``product'' and ``purchase size'' column to
Product\_name and Units\_purchased respectively for better understanding
of underlying data in the
column.}\label{lets-rename-the-product-and-purchase-size-column-to-product_name-and-units_purchased-respectively-for-better-understanding-of-underlying-data-in-the-column.}}

\begin{Shaded}
\begin{Highlighting}[]
\NormalTok{Store\_Transactions }\OtherTok{\textless{}{-}}\NormalTok{ Store\_Transactions }\SpecialCharTok{\%\textgreater{}\%}
  \FunctionTok{rename}\NormalTok{(}\AttributeTok{product\_name=}\NormalTok{product) }\SpecialCharTok{\%\textgreater{}\%}
  \FunctionTok{rename}\NormalTok{(}\AttributeTok{units\_purchased=}\NormalTok{purchase\_size)}
\end{Highlighting}
\end{Shaded}

\hypertarget{to-highlight-column-names-more-clearly.-lets-capitalize-column-names}{%
\paragraph{To highlight column names more clearly. Let's capitalize
column
names}\label{to-highlight-column-names-more-clearly.-lets-capitalize-column-names}}

\begin{Shaded}
\begin{Highlighting}[]
\NormalTok{Store\_Transactions }\OtherTok{\textless{}{-}} \FunctionTok{rename\_with}\NormalTok{(Store\_Transactions, toupper)}
\end{Highlighting}
\end{Shaded}

\hypertarget{lets-preview-to-see-if-the-changes-occured}{%
\paragraph{Let's preview to see if the changes
occured}\label{lets-preview-to-see-if-the-changes-occured}}

\begin{Shaded}
\begin{Highlighting}[]
\FunctionTok{head}\NormalTok{(Store\_Transactions)}
\end{Highlighting}
\end{Shaded}

\begin{verbatim}
##              DATE TRANSACTION_ID CUSTOMER_ID PRODUCT_NAME PRODUCT_CODE
## 1 29/08/2020 0:00           9900        1980          fan     SKU83503
## 2 01/05/2020 0:00          12315        2463          fan     SKU83503
## 3 12/12/2020 0:00           9890        1978          fan     SKU83503
## 4 16/02/2020 0:00          46915        9383          fan     SKU83503
## 5 28/12/2020 0:00          44700        8940          fan     SKU83503
## 6 28/12/2020 0:00          44700        8940         lamp     SKU95363
##   PRODUCT_COLOR PRODUCT_PRICE UNITS_PURCHASED PURCHASE_PRICE  REVENUE
## 1         brass         13.99               2         27.980  $27.98 
## 2         brass         13.99               2         27.980  $27.98 
## 3         white         13.99               1         13.990  $13.99 
## 4         black         13.99               1         13.990  $13.99 
## 5         brass         13.99               2         27.980  $27.98 
## 6         brass         45.99               5        160.965 $229.95
\end{verbatim}

\hypertarget{lets-load-another-package-to-make-changes-related-to-date}{%
\paragraph{Let's load another package to make changes related to
date}\label{lets-load-another-package-to-make-changes-related-to-date}}

\begin{Shaded}
\begin{Highlighting}[]
\FunctionTok{library}\NormalTok{(}\StringTok{"lubridate"}\NormalTok{)}
\end{Highlighting}
\end{Shaded}

\hypertarget{lets-see-the-format-type-of-date-column}{%
\paragraph{let's see the format type of ``date''
column}\label{lets-see-the-format-type-of-date-column}}

\begin{Shaded}
\begin{Highlighting}[]
\FunctionTok{class}\NormalTok{(Store\_Transactions}\SpecialCharTok{$}\NormalTok{DATE)}
\end{Highlighting}
\end{Shaded}

\begin{verbatim}
## [1] "character"
\end{verbatim}

\hypertarget{thus-to-be-able-to-perform-operations-on-the-date-letsconvert-date-from-char-to-date}{%
\paragraph{\texorpdfstring{Thus, to be able to perform operations on the
date lets'convert \textbf{date from char to
date}}{Thus, to be able to perform operations on the date lets'convert date from char to date}}\label{thus-to-be-able-to-perform-operations-on-the-date-letsconvert-date-from-char-to-date}}

\begin{Shaded}
\begin{Highlighting}[]
\NormalTok{Store\_Transactions}\SpecialCharTok{$}\NormalTok{DATE }\OtherTok{\textless{}{-}} \FunctionTok{ymd}\NormalTok{(Store\_Transactions}\SpecialCharTok{$}\NormalTok{DATE)}
\end{Highlighting}
\end{Shaded}

\begin{verbatim}
## Warning: All formats failed to parse. No formats found.
\end{verbatim}

\hypertarget{now-lets-see-if-the-change-has-occured}{%
\paragraph{Now, lets see if the change has
occured}\label{now-lets-see-if-the-change-has-occured}}

\begin{Shaded}
\begin{Highlighting}[]
\FunctionTok{class}\NormalTok{(Store\_Transactions}\SpecialCharTok{$}\NormalTok{DATE)}
\end{Highlighting}
\end{Shaded}

\begin{verbatim}
## [1] "Date"
\end{verbatim}

\hypertarget{now-we-will-remove-all-rows-with-n.a-values-in-columns.-otherwise-they-would-cause-problem-while-analysing-data.}{%
\paragraph{Now, we will remove all rows with N.A values in columns.
Otherwise, they would cause problem while analysing
data.}\label{now-we-will-remove-all-rows-with-n.a-values-in-columns.-otherwise-they-would-cause-problem-while-analysing-data.}}

\hypertarget{we-will-save-the-results-in-new-table-as-store_transaction}{%
\paragraph{We will save the results in new table, as
Store\_Transaction}\label{we-will-save-the-results-in-new-table-as-store_transaction}}

\begin{Shaded}
\begin{Highlighting}[]
\NormalTok{Store\_Transaction }\OtherTok{\textless{}{-}}\NormalTok{ Store\_Transactions[}\SpecialCharTok{!}\FunctionTok{is.na}\NormalTok{(Store\_Transactions}\SpecialCharTok{$}\NormalTok{PRODUCT\_NAME), ]}
\end{Highlighting}
\end{Shaded}

\hypertarget{or-in-the-code-we-can-also-mention-particular-rows-we-want-to-remove.}{%
\paragraph{OR In the code, we can also mention particular rows we want
to
remove.}\label{or-in-the-code-we-can-also-mention-particular-rows-we-want-to-remove.}}

\begin{Shaded}
\begin{Highlighting}[]
\NormalTok{Store\_Transaction }\OtherTok{\textless{}{-}}\NormalTok{ Store\_Transactions[}\SpecialCharTok{{-}}\FunctionTok{c}\NormalTok{(}\DecValTok{28}\NormalTok{,}\DecValTok{29}\NormalTok{),]}
\end{Highlighting}
\end{Shaded}

\hypertarget{lets-create-another-column-new_revenue-to-calculate-revenue-of-each-transaction-and-cross-check-it-with-column-named-purchase_price}{%
\paragraph{Let's create another column ``NEW\_REVENUE'' to calculate
revenue of each transaction and cross check it with column named
``PURCHASE\_PRICE''}\label{lets-create-another-column-new_revenue-to-calculate-revenue-of-each-transaction-and-cross-check-it-with-column-named-purchase_price}}

\begin{Shaded}
\begin{Highlighting}[]
\NormalTok{Store\_Transaction }\OtherTok{\textless{}{-}}\NormalTok{ Store\_Transaction }\SpecialCharTok{\%\textgreater{}\%} \FunctionTok{mutate}\NormalTok{(Store\_Transaction, }\AttributeTok{NEW\_REVENUE=}\NormalTok{ PRODUCT\_PRICE}\SpecialCharTok{*}\NormalTok{UNITS\_PURCHASED)}
\end{Highlighting}
\end{Shaded}

\hypertarget{now-we-will-remove-all-columns-that-we-dont-require-for-our-analysis.}{%
\paragraph{Now, we will remove all columns that we don't require for our
analysis.}\label{now-we-will-remove-all-columns-that-we-dont-require-for-our-analysis.}}

\hypertarget{we-will-also-be-removing-purchase-price-column-as-we-have-newly-created-accurate-column-named-new_revenue-in-place-of-it}{%
\paragraph{We will also be removing ``purchase price'' column as we have
newly created accurate column named ``new\_revenue'' in place of
it}\label{we-will-also-be-removing-purchase-price-column-as-we-have-newly-created-accurate-column-named-new_revenue-in-place-of-it}}

\begin{Shaded}
\begin{Highlighting}[]
\NormalTok{Store\_Transaction }\OtherTok{\textless{}{-}}\NormalTok{ Store\_Transaction }\SpecialCharTok{\%\textgreater{}\%} \FunctionTok{select}\NormalTok{(}\SpecialCharTok{{-}}\NormalTok{DATE,}\SpecialCharTok{{-}}\NormalTok{PRODUCT\_CODE,}\SpecialCharTok{{-}}\NormalTok{PURCHASE\_PRICE)}
\end{Highlighting}
\end{Shaded}

\hypertarget{now-lets-check-again-if-the-changes-we-made-occured-or-not}{%
\paragraph{Now, let's check again if the changes we made occured or
not}\label{now-lets-check-again-if-the-changes-we-made-occured-or-not}}

\begin{Shaded}
\begin{Highlighting}[]
\FunctionTok{head}\NormalTok{(Store\_Transaction)}
\end{Highlighting}
\end{Shaded}

\begin{verbatim}
##   TRANSACTION_ID CUSTOMER_ID PRODUCT_NAME PRODUCT_COLOR PRODUCT_PRICE
## 1           9900        1980          fan         brass         13.99
## 2          12315        2463          fan         brass         13.99
## 3           9890        1978          fan         white         13.99
## 4          46915        9383          fan         black         13.99
## 5          44700        8940          fan         brass         13.99
## 6          44700        8940         lamp         brass         45.99
##   UNITS_PURCHASED  REVENUE NEW_REVENUE
## 1               2  $27.98        27.98
## 2               2  $27.98        27.98
## 3               1  $13.99        13.99
## 4               1  $13.99        13.99
## 5               2  $27.98        27.98
## 6               5 $229.95       229.95
\end{verbatim}

\hypertarget{analysis-phase}{%
\subsection{ANALYSIS PHASE}\label{analysis-phase}}

\hypertarget{its-time-for-us-to-analyse-the-data-and-find-what-insights-we-can-get-from-it.}{%
\subsubsection{It's time for us to analyse the data and find what
insights we can get from
it.}\label{its-time-for-us-to-analyse-the-data-and-find-what-insights-we-can-get-from-it.}}

\hypertarget{every-transformation-we-will-make-in-orignal-dataset-to-pull-out-insights-we-will-be-saving-those-transformations-in-new-tables-in-order-to-make-visuals-from-them-later.}{%
\paragraph{Every transformation we will make in orignal dataset to pull
out insights, we will be saving those transformations in new tables in
order to make visuals from them
later.}\label{every-transformation-we-will-make-in-orignal-dataset-to-pull-out-insights-we-will-be-saving-those-transformations-in-new-tables-in-order-to-make-visuals-from-them-later.}}

\hypertarget{first-we-will-find-out-how-much-revenue-each-product-generated}{%
\paragraph{First, we will find out how much revenue each product
generated}\label{first-we-will-find-out-how-much-revenue-each-product-generated}}

\begin{Shaded}
\begin{Highlighting}[]
\CommentTok{\# Grouping and summarizing in order to find total Revenue generated from each product}
\NormalTok{Products\_vs\_Revenue }\OtherTok{\textless{}{-}}\NormalTok{ Store\_Transaction }\SpecialCharTok{\%\textgreater{}\%} \FunctionTok{group\_by}\NormalTok{(PRODUCT\_NAME) }\SpecialCharTok{\%\textgreater{}\%}
  \FunctionTok{summarize}\NormalTok{(}\AttributeTok{Total\_revenue\_of\_each\_product =} \FunctionTok{sum}\NormalTok{(NEW\_REVENUE)) }
\FunctionTok{head}\NormalTok{(Products\_vs\_Revenue)}
\end{Highlighting}
\end{Shaded}

\begin{verbatim}
## # A tibble: 6 x 2
##   PRODUCT_NAME Total_revenue_of_each_product
##   <chr>                                <dbl>
## 1 bed                                  800. 
## 2 bookcase                              58.9
## 3 chair                                234. 
## 4 couch                               9000  
## 5 desk                                 510. 
## 6 fan                                  112.
\end{verbatim}

\hypertarget{now-we-will-see-how-many-units-of-each-product-were-sold.}{%
\paragraph{Now, we will see how many units of each product were
sold.}\label{now-we-will-see-how-many-units-of-each-product-were-sold.}}

\begin{Shaded}
\begin{Highlighting}[]
\CommentTok{\# Grouping and summarizing in order to find how many units of each product were sold.}
\NormalTok{Products\_vs\_units }\OtherTok{\textless{}{-}}\NormalTok{ Store\_Transaction }\SpecialCharTok{\%\textgreater{}\%} \FunctionTok{group\_by}\NormalTok{(PRODUCT\_NAME) }\SpecialCharTok{\%\textgreater{}\%}
  \FunctionTok{summarize}\NormalTok{(}\AttributeTok{Total\_units\_sold\_of\_each\_product =} \FunctionTok{sum}\NormalTok{(UNITS\_PURCHASED))}
\FunctionTok{head}\NormalTok{(Products\_vs\_units)}
\end{Highlighting}
\end{Shaded}

\begin{verbatim}
## # A tibble: 6 x 2
##   PRODUCT_NAME Total_units_sold_of_each_product
##   <chr>                                   <int>
## 1 bed                                         1
## 2 bookcase                                    1
## 3 chair                                       1
## 4 couch                                       9
## 5 desk                                        3
## 6 fan                                         8
\end{verbatim}

\hypertarget{now-lets-see-the-revenue-generated-from-each-customer}{%
\paragraph{Now, let's see the revenue generated from each
customer}\label{now-lets-see-the-revenue-generated-from-each-customer}}

\begin{Shaded}
\begin{Highlighting}[]
\CommentTok{\# Grouping and summarizing in order to find total revenue generated from each customer}
\NormalTok{Customer\_vs\_revenue }\OtherTok{\textless{}{-}}\NormalTok{ Store\_Transaction }\SpecialCharTok{\%\textgreater{}\%} \FunctionTok{group\_by}\NormalTok{(CUSTOMER\_ID) }\SpecialCharTok{\%\textgreater{}\%}
  \FunctionTok{summarize}\NormalTok{(}\AttributeTok{Total\_revenue\_by\_each\_customer =} \FunctionTok{sum}\NormalTok{(NEW\_REVENUE))}
\FunctionTok{head}\NormalTok{(Customer\_vs\_revenue)}
\end{Highlighting}
\end{Shaded}

\begin{verbatim}
## # A tibble: 6 x 2
##   CUSTOMER_ID Total_revenue_by_each_customer
##         <int>                          <dbl>
## 1         335                         1000  
## 2        1268                          170. 
## 3        1928                         1000  
## 4        1978                           14.0
## 5        1980                         1028. 
## 6        2463                           28.0
\end{verbatim}

\hypertarget{now-we-will-see-number-of-units-bought-by-each-customer.}{%
\paragraph{Now, we will see number of units bought by each
customer.}\label{now-we-will-see-number-of-units-bought-by-each-customer.}}

\begin{Shaded}
\begin{Highlighting}[]
\CommentTok{\# Grouping and summarizing in order to find total units bought by each customer }
\NormalTok{Customer\_vs\_units\_purchased }\OtherTok{\textless{}{-}}\NormalTok{ Store\_Transaction }\SpecialCharTok{\%\textgreater{}\%} \FunctionTok{group\_by}\NormalTok{(CUSTOMER\_ID) }\SpecialCharTok{\%\textgreater{}\%}
  \FunctionTok{summarize}\NormalTok{(}\AttributeTok{Total\_units\_bought\_by\_each\_customer =} \FunctionTok{sum}\NormalTok{(UNITS\_PURCHASED))}
\FunctionTok{head}\NormalTok{(Customer\_vs\_units\_purchased)}
\end{Highlighting}
\end{Shaded}

\begin{verbatim}
## # A tibble: 6 x 2
##   CUSTOMER_ID Total_units_bought_by_each_customer
##         <int>                               <int>
## 1         335                                   1
## 2        1268                                   1
## 3        1928                                   1
## 4        1978                                   1
## 5        1980                                   3
## 6        2463                                   2
\end{verbatim}

\hypertarget{now-we-will-analyse-revenue-from-individual-products-which-are-available-with-different-colours.}{%
\paragraph{Now, we will analyse revenue from individual products which
are available with different
colours.}\label{now-we-will-analyse-revenue-from-individual-products-which-are-available-with-different-colours.}}

\hypertarget{first-lets-see-which-colour-of-product-fan-made-the-most-revenue}{%
\paragraph{First, let's see which colour of product ``Fan'' made the
most
revenue}\label{first-lets-see-which-colour-of-product-fan-made-the-most-revenue}}

\begin{Shaded}
\begin{Highlighting}[]
\CommentTok{\# Filtering to pull out products named "FAN"}
\NormalTok{PRODUCT\_FAN }\OtherTok{\textless{}{-}}\NormalTok{ Store\_Transaction }\SpecialCharTok{\%\textgreater{}\%} \FunctionTok{filter}\NormalTok{(PRODUCT\_NAME}\SpecialCharTok{==}\StringTok{\textquotesingle{}fan\textquotesingle{}}\NormalTok{)}
\CommentTok{\# Creating a new column by uniting 2 columns.}
\NormalTok{PRODUCT\_FAN }\OtherTok{\textless{}{-}} \FunctionTok{unite}\NormalTok{(PRODUCT\_FAN,}\StringTok{\textquotesingle{}PRODUCT\_NAME\_and\_COLOR\textquotesingle{}}\NormalTok{, PRODUCT\_NAME,PRODUCT\_COLOR, }\AttributeTok{sep =} \StringTok{\textquotesingle{} \textquotesingle{}}\NormalTok{)}
\CommentTok{\# Grouping and summarizing in order to find revenue of product generated by each of its colour variations}
\NormalTok{PRODUCT\_FAN }\OtherTok{\textless{}{-}}\NormalTok{ PRODUCT\_FAN }\SpecialCharTok{\%\textgreater{}\%} \FunctionTok{group\_by}\NormalTok{(PRODUCT\_NAME\_and\_COLOR) }\SpecialCharTok{\%\textgreater{}\%}
  \FunctionTok{summarize}\NormalTok{(}\AttributeTok{Total\_revenue\_by\_each\_color =} \FunctionTok{sum}\NormalTok{(NEW\_REVENUE))}
\end{Highlighting}
\end{Shaded}

\begin{Shaded}
\begin{Highlighting}[]
\FunctionTok{head}\NormalTok{(PRODUCT\_FAN)}
\end{Highlighting}
\end{Shaded}

\begin{verbatim}
## # A tibble: 3 x 2
##   PRODUCT_NAME_and_COLOR Total_revenue_by_each_color
##   <chr>                                        <dbl>
## 1 fan black                                     14.0
## 2 fan brass                                     83.9
## 3 fan white                                     14.0
\end{verbatim}

\hypertarget{now-lets-see-which-colour-of-product-couch-made-the-most-revenue}{%
\paragraph{Now, let's see which colour of product ``Couch'' made the
most
revenue}\label{now-lets-see-which-colour-of-product-couch-made-the-most-revenue}}

\begin{Shaded}
\begin{Highlighting}[]
\CommentTok{\# Filtering to pull out products named "COUCH"}
\NormalTok{PRODUCT\_COUCH }\OtherTok{\textless{}{-}}\NormalTok{ Store\_Transaction }\SpecialCharTok{\%\textgreater{}\%} \FunctionTok{filter}\NormalTok{(PRODUCT\_NAME}\SpecialCharTok{==}\StringTok{\textquotesingle{}couch\textquotesingle{}}\NormalTok{)}
\CommentTok{\# Creating a new column by uniting 2 columns.}
\NormalTok{PRODUCT\_COUCH }\OtherTok{\textless{}{-}} \FunctionTok{unite}\NormalTok{(PRODUCT\_COUCH,}\StringTok{\textquotesingle{}PRODUCT\_NAME\_and\_COLOR\textquotesingle{}}\NormalTok{, PRODUCT\_NAME,PRODUCT\_COLOR, }\AttributeTok{sep =} \StringTok{\textquotesingle{} \textquotesingle{}}\NormalTok{)}
\CommentTok{\# Grouping and summarizing in order to find revenue of product generated by each of its colour variations}
\NormalTok{PRODUCT\_COUCH }\OtherTok{\textless{}{-}}\NormalTok{ PRODUCT\_COUCH }\SpecialCharTok{\%\textgreater{}\%} \FunctionTok{group\_by}\NormalTok{(PRODUCT\_NAME\_and\_COLOR) }\SpecialCharTok{\%\textgreater{}\%}
  \FunctionTok{summarize}\NormalTok{(}\AttributeTok{Total\_revenue\_by\_each\_color =} \FunctionTok{sum}\NormalTok{(NEW\_REVENUE))}
\end{Highlighting}
\end{Shaded}

\begin{Shaded}
\begin{Highlighting}[]
\FunctionTok{head}\NormalTok{(PRODUCT\_COUCH)}
\end{Highlighting}
\end{Shaded}

\begin{verbatim}
## # A tibble: 6 x 2
##   PRODUCT_NAME_and_COLOR Total_revenue_by_each_color
##   <chr>                                        <dbl>
## 1 couch black                                   1000
## 2 couch blue                                    1000
## 3 couch brown                                   1000
## 4 couch grey                                    3000
## 5 couch purple                                  1000
## 6 couch white                                   2000
\end{verbatim}

\hypertarget{now-lets-see-which-colour-of-product-rug-made-the-most-revenue}{%
\paragraph{Now, let's see which colour of product ``Rug'' made the most
revenue}\label{now-lets-see-which-colour-of-product-rug-made-the-most-revenue}}

\begin{Shaded}
\begin{Highlighting}[]
\CommentTok{\# Filtering to pull out products named "RUG"}
\NormalTok{PRODUCT\_RUG }\OtherTok{\textless{}{-}}\NormalTok{ Store\_Transaction }\SpecialCharTok{\%\textgreater{}\%} \FunctionTok{filter}\NormalTok{(PRODUCT\_NAME}\SpecialCharTok{==}\StringTok{\textquotesingle{}rug\textquotesingle{}}\NormalTok{)}
\CommentTok{\# Creating a new column by uniting 2 columns.}
\NormalTok{PRODUCT\_RUG }\OtherTok{\textless{}{-}} \FunctionTok{unite}\NormalTok{(PRODUCT\_RUG,}\StringTok{\textquotesingle{}PRODUCT\_NAME\_and\_COLOR\textquotesingle{}}\NormalTok{, PRODUCT\_NAME,PRODUCT\_COLOR, }\AttributeTok{sep =} \StringTok{\textquotesingle{} \textquotesingle{}}\NormalTok{)}
\CommentTok{\# Grouping and summarizing in order to find revenue of product generated by each of its colour variations}
\NormalTok{PRODUCT\_RUG }\OtherTok{\textless{}{-}}\NormalTok{ PRODUCT\_RUG }\SpecialCharTok{\%\textgreater{}\%} \FunctionTok{group\_by}\NormalTok{(PRODUCT\_NAME\_and\_COLOR) }\SpecialCharTok{\%\textgreater{}\%}
  \FunctionTok{summarize}\NormalTok{(}\AttributeTok{Total\_revenue\_by\_each\_color =} \FunctionTok{sum}\NormalTok{(NEW\_REVENUE))}
\end{Highlighting}
\end{Shaded}

\begin{Shaded}
\begin{Highlighting}[]
\FunctionTok{head}\NormalTok{(PRODUCT\_RUG)}
\end{Highlighting}
\end{Shaded}

\begin{verbatim}
## # A tibble: 2 x 2
##   PRODUCT_NAME_and_COLOR Total_revenue_by_each_color
##   <chr>                                        <dbl>
## 1 rug beige                                     539.
## 2 rug grey                                      270.
\end{verbatim}

\hypertarget{now-lets-see-which-colour-of-product-desk-made-the-most-revenue}{%
\paragraph{Now, let's see which colour of product ``Desk'' made the most
revenue}\label{now-lets-see-which-colour-of-product-desk-made-the-most-revenue}}

\begin{Shaded}
\begin{Highlighting}[]
\CommentTok{\# Filtering to pull out products named "DESK"}
\NormalTok{PRODUCT\_DESK }\OtherTok{\textless{}{-}}\NormalTok{ Store\_Transaction }\SpecialCharTok{\%\textgreater{}\%} \FunctionTok{filter}\NormalTok{(PRODUCT\_NAME}\SpecialCharTok{==}\StringTok{\textquotesingle{}desk\textquotesingle{}}\NormalTok{)}
\CommentTok{\# Creating a new column by uniting 2 columns.}
\NormalTok{PRODUCT\_DESK }\OtherTok{\textless{}{-}} \FunctionTok{unite}\NormalTok{(PRODUCT\_DESK,}\StringTok{\textquotesingle{}PRODUCT\_NAME\_and\_COLOR\textquotesingle{}}\NormalTok{, PRODUCT\_NAME,PRODUCT\_COLOR, }\AttributeTok{sep =} \StringTok{\textquotesingle{} \textquotesingle{}}\NormalTok{)}
\CommentTok{\# Grouping and summarizing in order to find revenue of product generated by each of its colour variations}
\NormalTok{PRODUCT\_DESK }\OtherTok{\textless{}{-}}\NormalTok{ PRODUCT\_DESK }\SpecialCharTok{\%\textgreater{}\%} \FunctionTok{group\_by}\NormalTok{(PRODUCT\_NAME\_and\_COLOR) }\SpecialCharTok{\%\textgreater{}\%}
  \FunctionTok{summarize}\NormalTok{(}\AttributeTok{Total\_revenue\_by\_each\_color =} \FunctionTok{sum}\NormalTok{(NEW\_REVENUE))}
\end{Highlighting}
\end{Shaded}

\begin{Shaded}
\begin{Highlighting}[]
\FunctionTok{head}\NormalTok{(PRODUCT\_DESK)}
\end{Highlighting}
\end{Shaded}

\begin{verbatim}
## # A tibble: 2 x 2
##   PRODUCT_NAME_and_COLOR Total_revenue_by_each_color
##   <chr>                                        <dbl>
## 1 desk brown                                    340.
## 2 desk white                                    170.
\end{verbatim}

\hypertarget{share-phase}{%
\subsection{SHARE PHASE}\label{share-phase}}

\hypertarget{in-this-phase-we-will-present-the-insights-we-found-from-our-analysis-by-using-visualisations.}{%
\paragraph{In this phase, we will present the insights we found from our
analysis by using
visualisations.}\label{in-this-phase-we-will-present-the-insights-we-found-from-our-analysis-by-using-visualisations.}}

Note :\emph{I will be sharing the code for how to create visuals in
Rstudio. But, because they were difficult to understand for
stakeholder's, I will be sharing the visuals that I created using Google
sheets. They provide a accurate, detailed understanding of the insights
we pulled from data.}

\hypertarget{what-is-the-total-revenue-generated-by-each-product-1}{%
\subsubsection{1. What is the total revenue generated by each
product?}\label{what-is-the-total-revenue-generated-by-each-product-1}}

\begin{Shaded}
\begin{Highlighting}[]
\CommentTok{\# ggplot(data = Products\_vs\_Revenue) + }
\CommentTok{\#  geom\_bar(mapping =aes(x=Total\_revenue\_of\_each\_product, fill=PRODUCT\_NAME))}
\end{Highlighting}
\end{Shaded}

\begin{figure}
\centering
\includegraphics{/cloud/project/graphs/prr.png}
\caption{Fig.a}
\end{figure}

\hypertarget{its-surprising-to-see-that-the-product-couch-generated-the-most-revenue-for-our-store-as-compared-to-other-products.-the-revenue-is-literally-around-9000-while-we-couldnt-even-generate-minimum-2500-for-any-of-the-other-products.-this-possibly-has-multiple-reasons-such-as-we-sell-couches-with-the-most-variety-in-colors.-so-customers-prefer-to-buy-couch-from-our-store-as-there-are-many-varieties-available-with-respect-to-color.-another-reason-we-made-most-revenue-from-couch-is-because-its-also-the-most-expensive-product-in-our-furniture-shop-each-one-costing-1000.}{%
\paragraph{It's surprising to see that the product ``couch'' generated
the most revenue for our store as compared to other products. The
revenue is literally around 9000 \$, while we couldn't even generate
minimum 2500 \$ for any of the other products. This possibly has
multiple reasons such as, we sell couches with the most variety in
colors. So, customers prefer to buy couch from our store as there are
many varieties available with respect to color. Another reason we made
most revenue from ``couch'' is because it's also the most expensive
product in our furniture shop, each one costing
1000\$.}\label{its-surprising-to-see-that-the-product-couch-generated-the-most-revenue-for-our-store-as-compared-to-other-products.-the-revenue-is-literally-around-9000-while-we-couldnt-even-generate-minimum-2500-for-any-of-the-other-products.-this-possibly-has-multiple-reasons-such-as-we-sell-couches-with-the-most-variety-in-colors.-so-customers-prefer-to-buy-couch-from-our-store-as-there-are-many-varieties-available-with-respect-to-color.-another-reason-we-made-most-revenue-from-couch-is-because-its-also-the-most-expensive-product-in-our-furniture-shop-each-one-costing-1000.}}

\hypertarget{how-many-units-of-each-product-were-sold-1}{%
\subsubsection{2. How many units of each product were
sold?}\label{how-many-units-of-each-product-were-sold-1}}

\begin{Shaded}
\begin{Highlighting}[]
\CommentTok{\# ggplot(data = Products\_vs\_units) +}
\CommentTok{\#   geom\_bar(mapping = aes(x=PRODUCT\_NAME, fill=Total\_units\_sold\_of\_each\_product))}
\end{Highlighting}
\end{Shaded}

\begin{figure}
\centering
\includegraphics{/cloud/project/graphs/pus.png}
\caption{Fig.b}
\end{figure}

\hypertarget{its-clear-from-the-above-figure-that-the-total-units-sold-of-products-fan-rug-and-couch-are-highest-compared-to-other-products.-the-number-of-units-sold-of-this-products-were-minimum-8.-this-states-that-most-customers-are-in-need-of-fan-rug-couch-than-other-products.}{%
\paragraph{It's clear from the above figure that the total units sold of
products ``FAN, RUG and COUCH'' are highest compared to other products.
The number of units sold of this products were minimum 8. This states
that most customers are in need of FAN, RUG \& COUCH than other
products.}\label{its-clear-from-the-above-figure-that-the-total-units-sold-of-products-fan-rug-and-couch-are-highest-compared-to-other-products.-the-number-of-units-sold-of-this-products-were-minimum-8.-this-states-that-most-customers-are-in-need-of-fan-rug-couch-than-other-products.}}

\hypertarget{from-which-customer-have-we-made-the-most-revenue-1}{%
\subsubsection{3. From which customer have we made the most
revenue?}\label{from-which-customer-have-we-made-the-most-revenue-1}}

\begin{Shaded}
\begin{Highlighting}[]
\CommentTok{\# ggplot(data = Customer\_vs\_revenue) +}
\CommentTok{\#   geom\_bar(mapping = aes(x=CUSTOMER\_ID, fill=Total\_revenue\_by\_each\_customer))}
\end{Highlighting}
\end{Shaded}

\begin{figure}
\centering
\includegraphics{/cloud/project/graphs/crr.png}
\caption{Fig.c}
\end{figure}

\hypertarget{looking-at-this-graph-and-looking-back-to-our-earlier-findings-we-can-say-that-those-customers-who-bought-couches-from-our-store-generated-the-most-revenue-for-us-and-this-graph-indirectly-suggests-the-same.}{%
\paragraph{Looking at this graph and looking back to our earlier
findings, we can say that those customers who bought ``couches'' from
our store generated the most revenue for us and this graph indirectly
suggests the
same.}\label{looking-at-this-graph-and-looking-back-to-our-earlier-findings-we-can-say-that-those-customers-who-bought-couches-from-our-store-generated-the-most-revenue-for-us-and-this-graph-indirectly-suggests-the-same.}}

\hypertarget{how-many-products-did-each-customer-buy-1}{%
\subsubsection{4. How many products did each customer
buy?}\label{how-many-products-did-each-customer-buy-1}}

\begin{Shaded}
\begin{Highlighting}[]
\CommentTok{\# ggplot(data = Customer\_vs\_units\_purchased) +}
\CommentTok{\#   geom\_bar(mapping = aes(x=Total\_units\_bought\_by\_each\_customer , fill=PRODUCT\_NAME))}
\end{Highlighting}
\end{Shaded}

\begin{figure}
\centering
\includegraphics{/cloud/project/graphs/cpu.png}
\caption{Fig.d}
\end{figure}

\hypertarget{the-customer-with-id-8940-purchased-the-highest-number-of-furniture-products-from-our-store.-and-the-customer-who-bought-2nd-highest-number-of-products-from-our-store-has-customer-id9080.}{%
\paragraph{The customer with ID 8940 purchased the highest number of
furniture products from our store. And the customer who bought 2nd
highest number of products from our store has customer
ID9080.}\label{the-customer-with-id-8940-purchased-the-highest-number-of-furniture-products-from-our-store.-and-the-customer-who-bought-2nd-highest-number-of-products-from-our-store-has-customer-id9080.}}

\hypertarget{then-there-are-three-customers-who-bought-approximately-3-products-from-our-store-and-some-other-two-customers-bought-approximately-2-products-from-our-store.-remaining-customers-have-only-bought-1-product-from-our-store.}{%
\paragraph{Then there are three customers who bought approximately 3
products from our store and some other two customers bought
approximately 2 products from our store. Remaining customers have only
bought 1 product from our
store.}\label{then-there-are-three-customers-who-bought-approximately-3-products-from-our-store-and-some-other-two-customers-bought-approximately-2-products-from-our-store.-remaining-customers-have-only-bought-1-product-from-our-store.}}

\hypertarget{we-can-conclude-that-the-top-2-customers-who-bought-most-products-from-our-store-are}{%
\paragraph{We can conclude that the top 2 customers who bought most
products from our store
are}\label{we-can-conclude-that-the-top-2-customers-who-bought-most-products-from-our-store-are}}

\textbf{• ID8940} \textbf{• ID9080}

\hypertarget{which-color-is-most-preferred-by-customers-in-product-named-fan-1}{%
\subsubsection{5. Which color is most preferred by customers in product
named
``Fan''?}\label{which-color-is-most-preferred-by-customers-in-product-named-fan-1}}

\begin{Shaded}
\begin{Highlighting}[]
\CommentTok{\# ggplot(data = PRODUCT\_FAN) +}
\CommentTok{\#   geom\_bar(mapping = aes(x=Total\_revenue\_by\_each\_color, fill=PRODUCT\_NAME\_and\_COLOR))}
\end{Highlighting}
\end{Shaded}

\begin{figure}
\centering
\includegraphics{/cloud/project/graphs/fr.png}
\caption{Fig.e}
\end{figure}

\hypertarget{as-we-can-see-the-brass-colour-of-product-fan-is-more-preferred-by-customers-and-thus-has-generated-revenue-of-above-75-for-our-store.-while-the-white-black-colour-of-it-generated-comparatively-less-revenue-which-is-under-25.}{%
\paragraph{As we can see, the brass colour of product ``FAN'' is more
preferred by customers and thus has generated revenue of above 75 \$ for
our Store. While the white \& black colour of it generated comparatively
less revenue which is under
25\$.}\label{as-we-can-see-the-brass-colour-of-product-fan-is-more-preferred-by-customers-and-thus-has-generated-revenue-of-above-75-for-our-store.-while-the-white-black-colour-of-it-generated-comparatively-less-revenue-which-is-under-25.}}

\hypertarget{its-good-to-remember-that-all-colour-variants-of-this-product-are-sold-at-the-same-price.-but-because-the-brass-colour-variant-was-sold-more.-thus-it-generated-more-revenue-for-our-store.}{%
\paragraph{It's good to remember that all colour variants of this
product are sold at the same price. But, because the `brass' colour
variant was sold more. Thus, it generated more revenue for our
store.}\label{its-good-to-remember-that-all-colour-variants-of-this-product-are-sold-at-the-same-price.-but-because-the-brass-colour-variant-was-sold-more.-thus-it-generated-more-revenue-for-our-store.}}

\hypertarget{which-color-is-most-preferred-by-customers-in-product-named-couch-1}{%
\subsubsection{6. Which color is most preferred by customers in product
named
``Couch''?}\label{which-color-is-most-preferred-by-customers-in-product-named-couch-1}}

\begin{Shaded}
\begin{Highlighting}[]
\CommentTok{\# ggplot(data = PRODUCT\_COUCH) +}
\CommentTok{\#   geom\_bar(mapping = aes(x=Total\_revenue\_by\_each\_color, fill=PRODUCT\_NAME\_and\_COLOR))}
\end{Highlighting}
\end{Shaded}

\begin{figure}
\centering
\includegraphics{/cloud/project/graphs/cr.png}
\caption{Fig.f}
\end{figure}

\hypertarget{as-we-can-see-the-grey-colour-of-product-couch-is-more-preferred-by-customers-and-thus-has-generated-revenue-of-around-3000-for-our-store.-while-the-white-colour-of-it-made-comparatively-less-which-is-around-2000.}{%
\paragraph{As we can see, the Grey colour of product ``COUCH'' is more
preferred by customers and thus has generated revenue of around 3000 \$
for our Store. While the white colour of it made comparatively less
which is around
2000\$.}\label{as-we-can-see-the-grey-colour-of-product-couch-is-more-preferred-by-customers-and-thus-has-generated-revenue-of-around-3000-for-our-store.-while-the-white-colour-of-it-made-comparatively-less-which-is-around-2000.}}

\hypertarget{the-other-remaining-4-variants-generated-around-1000-each-for-our-store.}{%
\paragraph{The other remaining 4 variants generated around 1000\$ each
for our
store.}\label{the-other-remaining-4-variants-generated-around-1000-each-for-our-store.}}

\hypertarget{its-good-to-remember-that-all-colour-variants-of-this-product-are-sold-at-the-same-price.-but-because-the-grey-and-white-colour-variant-were-sold-more.-thus-they-generated-more-revenue-for-our-store.}{%
\paragraph{It's good to remember that all colour variants of this
product are sold at the same price. But, because the `Grey' and `White'
colour variant were sold more. Thus, they generated more revenue for our
store.}\label{its-good-to-remember-that-all-colour-variants-of-this-product-are-sold-at-the-same-price.-but-because-the-grey-and-white-colour-variant-were-sold-more.-thus-they-generated-more-revenue-for-our-store.}}

\hypertarget{which-color-is-most-preferred-by-customers-in-product-named-rug-1}{%
\subsubsection{7. Which color is most preferred by customers in product
named
``Rug''?}\label{which-color-is-most-preferred-by-customers-in-product-named-rug-1}}

\begin{Shaded}
\begin{Highlighting}[]
\CommentTok{\# ggplot(data = PRODUCT\_RUG) +}
\CommentTok{\#   geom\_bar(mapping = aes(x=Total\_revenue\_by\_each\_color, fill=PRODUCT\_NAME\_and\_COLOR))}
\end{Highlighting}
\end{Shaded}

\begin{figure}
\centering
\includegraphics{/cloud/project/graphs/rr.png}
\caption{Fig.g}
\end{figure}

\hypertarget{as-we-can-see-the-beige-colour-of-product-rug-is-more-preferred-by-customers-and-thus-has-generated-revenue-of-above-500-for-our-store.-while-the-grey-colour-of-it-generated-comparatively-less-revenue-which-is-around-300.}{%
\paragraph{As we can see, the beige colour of product ``RUG'' is more
preferred by customers and thus has generated revenue of above 500 \$
for our Store. While the grey colour of it generated comparatively less
revenue which is around
300\$.}\label{as-we-can-see-the-beige-colour-of-product-rug-is-more-preferred-by-customers-and-thus-has-generated-revenue-of-above-500-for-our-store.-while-the-grey-colour-of-it-generated-comparatively-less-revenue-which-is-around-300.}}

\hypertarget{its-good-to-remember-that-all-colour-variants-of-this-product-are-sold-at-the-same-price.-but-because-the-beige-colour-variant-was-sold-more.-thus-it-generated-more-revenue-for-our-store.}{%
\paragraph{It's good to remember that all colour variants of this
product are sold at the same price. But, because the `beige' colour
variant was sold more. Thus, it generated more revenue for our
store.}\label{its-good-to-remember-that-all-colour-variants-of-this-product-are-sold-at-the-same-price.-but-because-the-beige-colour-variant-was-sold-more.-thus-it-generated-more-revenue-for-our-store.}}

\hypertarget{which-color-is-most-preferred-by-customers-in-product-named-desk-1}{%
\subsubsection{8.Which color is most preferred by customers in product
named
``Desk''?}\label{which-color-is-most-preferred-by-customers-in-product-named-desk-1}}

\begin{Shaded}
\begin{Highlighting}[]
\CommentTok{\# ggplot(data = PRODUCT\_DESK) +}
\CommentTok{\#   geom\_bar(mapping = aes(x=Total\_revenue\_by\_each\_color, fill=PRODUCT\_NAME\_and\_COLOR))}
\end{Highlighting}
\end{Shaded}

\begin{figure}
\centering
\includegraphics{/cloud/project/graphs/dr.png}
\caption{Fig.h}
\end{figure}

\hypertarget{as-we-can-see-the-brown-colour-of-product-desk-is-more-preferred-by-customers-and-thus-has-generated-revenue-of-above-300-for-our-store.-while-the-white-colour-of-it-generated-comparatively-less-which-around-150.}{%
\paragraph{As we can see, the brown colour of product ``DESK'' is more
preferred by customers and thus has generated revenue of above 300 \$
for our Store. While the white colour of it generated comparatively less
which around
150\$.}\label{as-we-can-see-the-brown-colour-of-product-desk-is-more-preferred-by-customers-and-thus-has-generated-revenue-of-above-300-for-our-store.-while-the-white-colour-of-it-generated-comparatively-less-which-around-150.}}

\hypertarget{its-good-to-remember-that-all-colour-variants-of-this-product-are-sold-at-the-same-price.-but-because-the-brown-colour-variant-was-sold-more.-thus-it-generated-more-revenue-for-our-store.}{%
\paragraph{It's good to remember that all colour variants of this
product are sold at the same price. But, because the `brown' colour
variant was sold more. Thus, it generated more revenue for our
store.}\label{its-good-to-remember-that-all-colour-variants-of-this-product-are-sold-at-the-same-price.-but-because-the-brown-colour-variant-was-sold-more.-thus-it-generated-more-revenue-for-our-store.}}

\hypertarget{recommendations}{%
\subsection{Recommendations :}\label{recommendations}}

\hypertarget{fan-rug-couch-are-the-most-in-demand-product-so-we-should-ensure-that-theres-sufficient-stock-of-this-products-in-our-inventory.}{%
\paragraph{1. FAN, RUG, COUCH are the most in demand product, so we
should ensure that there's sufficient stock of this products in our
inventory.}\label{fan-rug-couch-are-the-most-in-demand-product-so-we-should-ensure-that-theres-sufficient-stock-of-this-products-in-our-inventory.}}

\hypertarget{we-have-2-most-loyal-customers-who-generally-buy-from-our-store.-so-from-time-to-time-we-should-see-if-they-are-in-need-of-any-furniture-and-provide-them-with-best-offers-for-being-a-loyal-customer-to-our-shop.-this-will-also-encourage-other-customers-to-fulfill-most-of-their-furniture-needs-from-our-store.}{%
\paragraph{2. We have 2 most loyal customers, who generally buy from our
store. So, from time to time we should see if they are in need of any
furniture and provide them with best offers for being a loyal customer
to our shop. This will also encourage other customers to fulfill most of
their furniture needs from our
store.}\label{we-have-2-most-loyal-customers-who-generally-buy-from-our-store.-so-from-time-to-time-we-should-see-if-they-are-in-need-of-any-furniture-and-provide-them-with-best-offers-for-being-a-loyal-customer-to-our-shop.-this-will-also-encourage-other-customers-to-fulfill-most-of-their-furniture-needs-from-our-store.}}

\hypertarget{we-should-keep-more-variants-of-every-single-product-as-people-want-to-choose-from-a-range-of-varieties.-also-we-should-try-to-keep-those-furniture-products-that-are-generally-expensive-as-they-will-generate-the-most-revenue-or-profit-for-us.}{%
\paragraph{3. We should keep more variants of every single product, as
people want to choose from a range of varieties. Also, we should try to
keep those furniture products that are generally expensive, as they will
generate the most revenue or profit for
us.}\label{we-should-keep-more-variants-of-every-single-product-as-people-want-to-choose-from-a-range-of-varieties.-also-we-should-try-to-keep-those-furniture-products-that-are-generally-expensive-as-they-will-generate-the-most-revenue-or-profit-for-us.}}

\hypertarget{currently-product-couch-is-generating-the-most-revenue-for-us.-so-its-important-to-ensure-that-couch-sales-continue-like-this-by-running-the-business-operations-for-product-couch-without-any-change-for-now.}{%
\paragraph{4. Currently, product ``Couch'' is generating the most
revenue for us. So, it's important to ensure that couch sales continue
like this by running the business operations for product ``couch''
without any change for
now.}\label{currently-product-couch-is-generating-the-most-revenue-for-us.-so-its-important-to-ensure-that-couch-sales-continue-like-this-by-running-the-business-operations-for-product-couch-without-any-change-for-now.}}

\hypertarget{as-seen-earlier-in-products-that-have-different-color-varieties.-certain-color-of-each-of-this-product-get-purchased-more-than-others.-so-we-should-maintain-their-stocks-in-our-inventory-as-they-are-more-preferred-color-variants.}{%
\paragraph{5. As seen earlier in products that have different color
varieties. Certain color of each of this product get purchased more than
others. So, we should maintain their stocks in our inventory as they are
more preferred color
variants.}\label{as-seen-earlier-in-products-that-have-different-color-varieties.-certain-color-of-each-of-this-product-get-purchased-more-than-others.-so-we-should-maintain-their-stocks-in-our-inventory-as-they-are-more-preferred-color-variants.}}

\hypertarget{in-short-they-are.}{%
\paragraph{In short, they are.}\label{in-short-they-are.}}

\hypertarget{for-couch-preferred-colours-are-grey-and-white.}{%
\paragraph{• For ``COUCH'' preferred colours are grey and
white.}\label{for-couch-preferred-colours-are-grey-and-white.}}

\hypertarget{for-rug-preferred-colour-is-beige.}{%
\paragraph{• For ``RUG'' preferred colour is
beige.}\label{for-rug-preferred-colour-is-beige.}}

\hypertarget{for-fan-preferred-colour-is-brass.}{%
\paragraph{• For ``FAN'' preferred colour is
brass.}\label{for-fan-preferred-colour-is-brass.}}

\hypertarget{for-desk-preferred-colour-is-brown.}{%
\paragraph{• For ``DESK'' preferred colour is
brown.}\label{for-desk-preferred-colour-is-brown.}}

\end{document}
